\chapter{Konzeption}
\label{Konzeption}
In erster Linie soll eine GWT Frontend Anwendung generiert werden, basierend auf
einer vorgegebenen Architektur (vgl. Abschnitt \ref{AufBZielArchitektur}). Dabei
ist eines der Hauptziele die leichte Erstellung der GWT Anwendung, ohne
Abhängigkeiten zu der zu generierenden Architektur. Darüber hinaus sollen
Vorteile durch vorgegebene Architekturpatterns wie der leichte Austausch von
View Implementierungen durch MVP weiterhin nutzbar sein. Zusätzlich ist es
wünschenswert, das Verhalten von View Komponenten wie Buttons beispielsweise für eine Navigation
zwischen den Webseiten zu ermöglichen. Weitere dieser Verhaltensspezifikationen
können das Öffnen eines Popups sowie die Übertragung von Daten sein. Auch hierbei
steht die einfache Erstellung einer GWT Anwendung und die Konformität
der Architektur im Vordergrund.

Es sollen alle von GWT vorgegebenen View Komponenten wie Label und
Menus für den Entwickler verwendbar sein, welche zusätzlich
untereinander zugeordnet werden können. Des Weiteren sollen Elemente, die auf
jeder Webseite zu sehen sind, wie z.B. ein Header implementiert werden. Layout-
und Stylevorgaben sollen nicht berücksichtigt werden, da dafür UI Editoren
existieren.

Es sollen weitere eigene View Komponenten, z.B. eine Datentabelle, die mehrmals
verwendet wird, erstellt werden können, damit, unabhängig der vorgegebenen
Architektur, weitere architektonische Maßnahmen erfolgen können. Darüber hinaus soll
eine vorgegebene sowie eine durch den Nutzer erstellte Paketierung
für Views generiert werden.

Zur Erstellung einer GWT Frontend Anwendung soll ein UML Modell auf M1 Ebene
erstellt werden (vgl. Abschnitt \ref{AufBM1}), basierend auf dem zum
Generator Projekt gehörendem UML Profil auf M2 Ebene (vgl. Abschnitt
\ref{AufBProfil}), welches durch den Generator (vgl. Abschnitt \ref{AufBGenerator}) mittels
MTL generiert werden soll.
% ---------------------------------------------
% Konzepte
% ---------------------------------------------
% ---------------------------------------------
% ZielArchitektur
% ---------------------------------------------
\section{Ziel-Architektur}\label{AufBZielArchitektur}
Die für die Generierung vorgesehene Architektur basiert auf 
Architekturkonzepten verschiedener Entwickler und entstand bei der Entwicklung
von vorhergehenden GWT Projekten. Diese Architektur stellte sich dabei als Best
Practice Lösung heraus, welche jedoch aufwändig und fehleranfällig bei der
Umsetzung ist. Dies ist einer der Gründe einen Generator für GWT Frontend
Anwendungen zu entwickeln. Im Folgendem wird die umzusetzende Architektur
kategorisiert und anhand der zu erstellenden Klassen und Dateien vorgestellt.
% -------------------Struktur Erklärungen--------------------------
\begin{itemize}
  \item einmalig vorhandene Dateien und Klassen
  \begin{itemize}
    \item \textbf{index.html}\\
    HTML Seite über die, durch GWT, die in Java erzeugten View Klassen
    eingebunden werden.
    \item \textbf{styles.css}\\
    CSS Datei für die Festlegung der Style-Eigenschaften.
    \item \textbf{\grqq{}Name\grqq{}.gwt.xml}\\
    Konfigurationsdatei in der u.A. verwendete Bibliotheken sowie
    Browsereinstellungen und die Zugriffsklasse eingetragen wird.
    \item \textbf{AppEntryPoint.java}\\
    Zugriffsklasse, d.h. die Einstiegsklasse für die GWT Anwendung. In dieser
    Klasse wird u.A. die ContentView sowie die PermanentViews und die
    HistoryMapper definiert.
    \item \textbf{AbstractView.java}\\
    Ist die Oberklasse aller View Klassen Implementierungen innerhalb der
    esrtellten GWT Anwendung. Diese definiert Eigenschaften für alle Views und
    ermöglicht als Oberklasse den Austausch der
    Ansichten.-----------------------------------
    \item \textbf{AbstractActivityDefaultImpl.java}\\
    Diese Klasse wird von allen View Activity Klassen erweitert und dient
    mittels \textit{start}-Methode dazu die View Klassen aufzurufen und dem
    Zugriff auf die Views über den Browser mittels \textit{Place}.---------
    \item \textbf{\grqq{}Name\grqq{}ViewActivityMapperImpl.java}\\
    Definiert die \textit{PlaceControllerProvider} damit darüber der View Place
    aufgerufen werden kann und somit im Browser die View Implementierung
    erscheint. Diese Klasse existiert prinzipiell einmal. Kommt jedoch eine
    PermanentView hinzu so kommt je PermanentView eine weitere
    ViewActivityMapper-Implementierung dazu, welche sich jeweils ihren
    PermanentViewActivity als \textit{PlaceControllerProvider}
    definiert. Diese \textit{PlaceControllerProvider} werden
    dann nicht in der ursprünglich existierenden
    ViewActivityMapper-Implementierung definiert.-----------------
    \item \textbf{AppPlaceHistoryMapper.java}\\
    Ermöglicht über die View Places den Zugriff auf die gezeigte View
    Implementierung durch den Back-Button im Browser oder innerhalb der View
    Implementierung nachdem dies in der View Activity implementiert wurde.
    \item \textbf{AppGinjector.java}\\
    Über diese Java Klasse wird es u.A. ermöglicht die
    ViewActivityMapper-Implementierungen sowie den EventBus zu
    erhalten, welcher die Navigations-Historie enthält bzw. speichert. Darüber
    können diese weiterhin in dem \textit{AppEntryPoint} definiert werden.
    \item \textbf{PlaceControllerProvider.java}\\
    Ist die Schnittstelle zu den View Places. Damit können diese in der
    ViewActivityMapper-Implementierung aufgerufen werden und dadurch der Zugriff
    auf die View Implementierungen zur Ansicht im Browser ermöglicht werden.
    Dies wird über die Einbindung von GIN ermöglicht.
    \item \textbf{ProductionGinModule.java}\\
    In dieser Klasse werden die für GIN typischen bind-Befehle definiert. Diese
    dienen u.A. dazu die View Interfaces an die View Implementierungen zu
    binden, sowie die Start-View festzulegen.
  \end{itemize}  
  \item View Klassen und Dateien, welche für jede View implementiert werden,
  basierend auf dem MVP-Pattern
    \begin{itemize}
    \item \textbf{\grqq{}Name\grqq{}Activity.java}\\
    Dient der Implementierung des Presenters sowie der Definierung der View.
    Darüber hinaus wird innerhalb dieser Klasse der PlaceController definiert,
    worüber u.A. eine Navigation zwischen den Webseiten im Browser erfolgen kann
    mittels einer \textit{goTo}-Methode.
    \item \textbf{\grqq{}Name\grqq{}Place.java}\\
    Diese Implementierungen sehen im groben inhaltlich immer gleich aus. Die
    Unterscheidung ist hierbei die dazugehörige View. Über diese Klasse wird die
    Navigation innerhalb der Seiten bzw. View Implementierungen ermöglicht.
    \item \textbf{\grqq{}Name\grqq{}View.java}\\
    Hierbei handelt es sich um ein Interface, welches des Presenter Interface
    enthält und ist die Oberklasse für die jeweiligen View Implementierungen.
    Diese Schnittstelle ermöglicht den einfachen Austausch der verschiedenen
    View Implementierungen für die Anwendung, welches zusätzlich über einen
    \textit{bind}-Befehl innerhalb des \textit{ProductionGinModule} festgelegt
    werden muss.
    \item \textbf{\grqq{}Name\grqq{}ViewImpl.java}\\
    Dabei handelt es sich um die konkreten View Implementierungen, welche im
    Browser sichtbar sind. Diese implementieren die jeweilige View und enthalten
    den durch die View und Activity implementierten Presenter, wodurch die
    Kontrolle der View Implementierung seitens des MVP-Prinzips abgegeben wird.
    Darüber hinaus kann diese Klasse die notwendigen View Komponenten
    definieren, wenn dies erforderlich, z.B. zum Befüllen von Datentabellen,
    oder dies gewünscht, z.B. zum Zugriff auf Buttons oder zur inhaltsabhängigen
    Erstellung von View Komponenten, ist. Die View Implementierung ist an
    eine bestimmte fest definierte
    \textit{\grqq{}Name\grqq{}ViewImpl.ui.xml}-Datei gebunden, durch die Angabe
    des gleichen Namens \grqq{\grqq{}Name\grqq{}ViewImpl}\grqq. Zusätzlich
    können durch die Annotation \textit{@UIField} die View Komponenten der
    \textit{\grqq{}Name\grqq{}ViewImpl.ui.xml}-Datei innerhalb der
    View Implementierung aufgerufen und definiert werden.
    \item \textbf{\grqq{}Name\grqq{}ViewImpl.ui.xml}\\
    Innherhalb dieser Datei können Style-Eigenschaften zu den View Komponenten
    sowie die View Komponenten definiert werden.-----------
  \end{itemize} 
  \item View Klassen und Elemente die auf jeder Ansicht zu sehen sind:
    \begin{itemize}
    \item werden auch nach dem MVP Pattern, wie oben beschrieben, erstellt.
    \item innerhalb des AppEntryPoint enthalten und definiert.
    \item bilden jeweils eine eigene
    \textit{\grqq{}Name\grqq{}ViewActivityMapper} Implementierung.
  \end{itemize} 
\end{itemize}

% -------------------View Eintragungen--------------------------
Basierend dieser Architektur muss zur Erzeugung einer View, diese eingetragen
werden innerhalb der der folgenden Klassen:
    \begin{itemize}
    \item \grqq{}Name\grqq{}ActivityMapperImpl.java
    \item AppPlaceHistoryMapper.java
    \item ProductionGinModule.java
  \end{itemize} 
und folgende Klassen und Dateien erzeugt werden:
   \begin{itemize}
    	\item \grqq{}Name\grqq{}Activity.java
    	\item \grqq{}Name\grqq{}Place.java
    	\item \grqq{}Name\grqq{}View.java
    	\item \grqq{}Name\grqq{}ViewImpl.java
    	\item \grqq{}Name\grqq{}ViewImpl.ui.xml
  \end{itemize} 
unter Betrachtung, dass ausschließlich eine View Implementierung existiert.
Existieren zu einer View mehrere View Implementierungen so müssen mehrere
Eintragungen innerhalb des \textit{ProductionGinModule.java} getätigt werden und
mehrere \textit{\grqq{}Name\grqq{}ViewImpl.java} und
\textit{\grqq{}Name\grqq{}ViewImpl.ui.xml} erstellt werden. Dadurch wird eine
gute Abstraktion und lose Kopplung geschaffen. Dies ist jedoch sehr
aufwändig und fehleranfällig, da leicht ein Eintrag vergessen werden kann und
viele Klassen erzeugt werden müssen.

% -------------------Packages--------------------------
Zu den genannten Architekturvorstellungen gehört zusätzlich eine Paketierung,
die auch durch vorherige GWT-Projekte entstand. Der Vorteil der folgenden
Gliederung der Pakete besteht darin, dass innherlab der
\textit{\grqq{}Name\grqq{}.gwt.xml} Konfigurationsdatei das source-Tag, welches
den Pfad für den zu übersetzenden Java Code angibt, wie folgt: $<source
path='client'/>$ definiert werden kann. Folgend die Gliederung der Klassen
innerhalb ihrer Packages:
\begin{itemize}
  \item \grqq{}projektname\grqq{}
  	\begin{itemize}
    	\item \grqq{}Name\grqq{}.gwt.xml
  	\end{itemize}
  \item \grqq{}projektname\grqq{}.client
    \begin{itemize}
    	\item AppEntryPoint.java
    \end{itemize}
  \item \grqq{}projektname\grqq{}.client.common
    \begin{itemize}
      	\item AbstractView.java
  	  	\item AbstractActivityDefaultImpl.java
    	\item \grqq{}Name\grqq{}ViewActivityMapperImpl.java
    	\item AppPlaceHistoryMapper.java
    \end{itemize}
  \item \grqq{}projektname\grqq{}.client.gin
    \begin{itemize}
   		\item AppGinjector.java
    	\item PlaceControllerProvider.java
    	\item ProductionGinModule.java
    \end{itemize}

  \item \grqq{}projektname\grqq{}.client.view
    \begin{itemize}
    	\item \grqq{}Name\grqq{}Activity.java
    	\item \grqq{}Name\grqq{}Place.java
    	\item \grqq{}Name\grqq{}View.java
    	\item \grqq{}Name\grqq{}ViewImpl.java
    	\item \grqq{}Name\grqq{}ViewImpl.ui.xml
	\end{itemize}
\end{itemize}
Jedoch soll für einen Entwickler die Möglichkeit bleiben innerhalb des view
Packages, die Views in Packages zu gliedern. Aus diesem Grund soll an dieser
Stelle das view Package nicht Generator-seitig tiefer gegliedert werden.

Anhand der beschrieben Architektur wird ersichtlich, dass die Erstellung eines
GWT Projektes hauptsächlich im Bereich der einmalig vorhanden Dateien und
Klassen und die Erstellung einer View im groben immer gleich ist. Dies bietet
zwar den Vorteil der Vereinheitlichung mehrerer GWT Projekte und gewährleistet
eine gewisse Übersichtlichkeit und kurze Einarbeitungszeit in verschiedenen GWT
Projekten, ist aber sehr aufwändig und fehleranfällig. Beispielsweise kann über
\grqq{}Copy-Paste\grqq{} viel esrtellt und implementiert werden, jedoch ist
dabei das Risiko erhöht, dass Einträge vergessen werden abzuändern oder
hinzuzufügen. Darüber hinaus kann es passieren, das Einträge enthalten sind wie
z.B. einer Bibliothek, welche jedoch nicht mehr benötigt werden und somit nicht
im Build-Path enthalten sind.
Diese Beispiele führen potenziell alle dazu, dass die gesamte GWT Anwendung nicht mehr startet
und die Suche nach dem Fehler erschwert wird, da oftmals viele dieser
Fehler flüchtig geschehen können. 
% ---------------------------------------------
% Profil
% ---------------------------------------------
\section{Profil}\label{AufBProfil}
Eines der Hauptziele ist wie erwähnt die erleichterte Erstellung von GWT
Projekten unter Nutzung der durch die Architektur gegebenen Vorteile. Zu welchen
auch der einfache Austausch von View Implementierungen unabhängig vom Model
gehört. Des Weiteren soll der Aufwand und die Fehleranfälligkeit bei der
Erstellung eines GWT Projektes sowie zur Erstellung von Views (vgl.
Abschnitt \ref{AufBZielArchitektur}) minimiert werden werden. Diese Ziele
erfordern einen hohen Abstraktionsgrad innerhalb des Profils sowie weiterhin des
M1-Modells.

Deswegen soll eine der ersten Überlegungen dazu führen, dass das
gesamte GWT Projekt auf ein gemeinsames Element reduziert werden soll. Dieses
Element erscheint im Rahmen des, aus der Architektur heraus, umzusetzenden
MVP-Patterns in dem die View Klassen und Dateien
\textit{\grqq{}Name\grqq{}Activity.java}, \textit{\grqq{}Name\grqq{}Place.java},
\textit{\grqq{}Name\grqq{}View.java}, \textit{\grqq{}Name\grqq{}ViewImpl.java}
und \textit{\grqq{}Name\grqq{}ViewImpl.ui.xml} das gemeinsame Element für alle
anderen Klassen und Inhalte bieten. Jedoch erschien dadurch der Aufwand bei der
Erstellung von Views nicht minimiert, da durch diese Stereotypenbildung, diese
im M1-Modell weiterhin Einsatz finden müssen. Aus diesem Grund ist eine weitere
Suche nach dem gemeinsamen Element erforderlich. Dies führt durch die, sich auch
in diesem Rahmen als vorteilhaft herausstellende, Architektur dazu, dass dies
das unterste Element ist, die \textbf{View Implementierung}.  Diese erscheint
ausreichend für die Erstellung der gesamten einmalig vorhandenen Klassen und
Dateien sowie der View Klassen und Dateien, da innerhalb der View
Implementierung und der simultanen und vereinheitlichten Namensbennenung alle
notwendigen Teile generierbar wären. 

Eine weitere Anforderung ist der Austausch der View Implementierungen durch den
Einsatz von MVP. Dieser kann jedoch nicht ausschließlich durch die View
Implementierung erfolgen, da hierbei eine Möglichkeit gegeben werden muss zu der
Verbindung meherer View Implementierungen zueinander. Aus diesem Grund soll das
\textbf{View Interfaces} genutzt werden. Dies stellt die Verbindung von mehreren
View Implementierungen in Form einer Oberklasse her. Zusätzlich bietet dies die
Möglichkeit bestimmte Methoden zu definieren, welche für jede implementierende
View nützlich ist. 

View Implementierungen haben zusätzlich View Komponenten und darüber hinaus ist
die Umsetzung einer Navigation bzw. das Verhalten bei Interaktion mit
View Komponenten ein weiteres umzusetzendes Ziel. In vorhergehenden Generator
Projekten ging die Umsetzung dessen mittels Enumerations hervor. Diese sind
sinnvoll wenn eine View Komponente mehere Attribute z.B. eine Value haben soll
und kein konkreter Nachbau von bestehenden Frameworks erfolgen soll. In dem Profil
wird ein hoher Abstraktionsgrad erwartet, damit eine Vereinfachung gewährleistet
werden kann. Deswegen bildet der Einsatz von Enumerations als Typ für
View Komponenten einen Mehrwert für die Umsetzung des Profils. Zusätzlich wird ein Layouting als
Ziel ausgeschlossen, wodurch die Nachteile der Verwendung von Enumerations
verringert werden. Dadurch ergibt sich zusätzlich der Einsatz von \textbf{View
Komponenten} als Stereotypen mit dem Attribut \textit{type} vom Typ
\textbf{Enumeration View Objekt Typ}.

Für Navigationselemente wie Buttons und Umsetzung der Navigation ist eine
Überlegung ein weiteres Profil für ActivityDiagramme und somit ein
ActivityDiagramm als M1 Modell zu nutzen. Dieses Mittel ermöglicht eine
Übersicht über die Navigationsstruktur bzw. Verhaltensstruktur einer Anwendung
und ist somit gut geeignet. Dennoch ist das Ziel der Vereinfachung nicht
gegeben, da für jedes Navigationselement ein extra Eintrag in einem
ActivityDiagramm erfolgen muss und somit eine Redundanz innerhalb der
verschiedenen M1-Modelle entsteht und der Mehrwert der Navigationsgenerierung
seitens des Entwicklers, welcher das Generator Projekt nutzen möchte, mindert
aufgrund des Mehraufwandes. Die Verwendung von Enumerations für View Objekte
ermöglichte die Idee, dass für navigierbare bzw. verhaltensbasierte Elemente
eine ähnliche Handhabung dienlich sein kann. Diese ermöglichen die Erfüllung der
Zielsetzung und bieten über die Angabe eines Attributes \textit{goTo} die
Navigation sowie über weiterer Attribute z.B. \textit{openPopup} eine
Generierung verhaltensbezogener Inhalte über das M1-Modell. Dies führt zu der
Stereotypenbildung der \textbf{Navigations View Komponenten} mit
\textbf{Enumeration Navigations View Objekt Typen}. Diese Enumeration enthält
dann z.B. Tree, Button oder MenuBar, d.h. View Komponenten von GWT, welche für
Navigationen oder Verhaltensspezifikationen vorgesehen sind.

\textbf{Navigations View Komponente} und \textbf{View Komponenten} sind
Properties, da eine View Implementierung diese View Komponenten enthalten kann.

Zur Kopplung von großen Views und zur Erweiterung der vorgegebenen Mittel anhand
der Architektur soll eine weitere Klasse als Stereotyp dienen, die
\textbf{eigenen View Komponenten}. Darin sollen bestehende View Komponenten
enthalten sein und somit eine eigene View Komponente, z.B. eine große
Datentabelle, bilden, die innerhalb einer View Implementierung enthalten sein
kann.

Im Bereich von Views z.B. ein Header die auf allen Views innerhalb des Browsers
sichtbar sind, soll eine weiterer Stereotypenbildung stattfinden. Diese Views
sind \textbf{Permanente Views} und gliedern sich zusätzlich in \textbf{Header}
und \textbf{Footer}, da diese konkrete permanente Views sind durch ihre
Positionierung in einer View (Header oben, Footer unten). Diese Views verhalten
sich simultan zu den normalen Views, weshalb die permanenten Views als View
Implementierung von dem View Interface umgesetzt werden können. Dadurch wird
zusätzlich der Austausch von View Implementierungen ermöglicht.
% ---------------------------------------------
% M1-Modell
% ---------------------------------------------
\section{M1-Modell}\label{AufBM1}
Basierend auf dem UML-Profil soll das Modell View Implementierungen enthalten,
welche wiederum View Komponenten als Properties beinhalten können sowie eigene
View Objekte zu diesen zugeordnet sein können. Darüber hinaus sollen über
Attribute innerhalb von Navigations View Komponenten, im M1-Modell enthaltene,
View Implementierungen angegeben werden, sodass z.B. über das Attribut goTo die
Navigation zu dieser View Implementierung durch Generierung erfolgen kann.
Darüber hinaus sollen permanente View Implementierungen in Form z.B. eines
Headers erstellt werden, welche dann auf jeder View innerhalb des Browsers
sichtbar sind. Eine Implementierung eines Headers soll eine MenuBar enthalten,
die durch MenuItems die Navigation zu verschiedenen Seiten ermöglicht.
Zusätzlich sollen, z.B. für die Implementierung einer GWT Anwendung, für
verschiedene Devices z.B. einen Browser auf einen Desktop oder einem Browser auf
einem Smartphone, View Intefaces rstellt werden, welche mehrere
Implementierungen haben, sodass diese für verschiedene Devices ausgelegt angezeigt werden.
% ---------------------------------------------
% Generator
% ---------------------------------------------
\section{Generator}\label{AufBGenerator}
Im Bereich des UML-Profils sowie des M1-Modells sind außer der Views und ihren
Implementierungen und ihren View Komponenten die weiteren Klassen und Dateien
ungeachtet geblieben. Aus diesem Grund muss der Generator so geschrieben werden,
sodass aus dem M1-Modell ein gesamtes GWT Projekt generiert werden kann. Dies
soll potenziell auch ermöglicht werden in dem nur eine View Implementierung
erstellt wird, welche ohne Attribute und Methoden ausgestattet ist. Dazu soll
eine Unterleitung erfolgen, welche einerseits die einmalig vorhandenen Klassen
mit Inhalten (auch View abhängiger Inhalte) generiert und andererseits die Views
generiert. Existieren View Interfaces so sollen die Klassen \textit{\grqq{}Interfacename\grqq{}View.java},
\textit{\grqq{}Interfacename\grqq{}Activity.java} und
\textit{\grqq{}Interfacename\grqq{}Place.java} den Interfacenamen tragen und die
Implementierungsklassen bzw. Dateien
\textit{\grqq{}Klassenname\grqq{}ViewImpl.java} und
\textit{\grqq{}Klassenname\grqq{}ViewImpl.ui.xml} der View, die
Implemntierungsklassennamen. Darüber hinaus sollen die View abhängigen Inhalte,
wenn mehr als eine View Implementierung vorhanden ist, mittels eines boolean
\textit{binding} dementsprechend unterschieden werden, sodass alle Inhalte
basierend auf View Implementierungen auskommentiert werden, bei binding gleich
false. Dadurch kann ein einfacher Austausch bei einem Wechsel
der View über das MVP-Pattern ermöglicht werden, ohne Code hinzufügen zu müssen.
In dem Fall, dass kein View Interface zu einer Implementierung gehört, so werden
alle Klassen, Dateien und View abhängige Inhalte mittels des
Implementierungsklassennamen generiert.

Für alle Permanenten Views gilt ein
ähnliches Vorgehen wie bei den normalen Views. Jedoch haben diese die
Besonderheit, dass sie in jeder View zu sehen sind und somit zusätzlich zu der
einen \textit{AbstractView} in dem \textit{AppEntryPoint.java} eingetragen
werden. Dazu erfolgt eine durch das Profil vorgegebene Unterscheidung zwischen
normalen permanenten Views und Header und Footer. Da Header (oben) und Footer
(unten) eine feste Positionierung auf einer Webseite haben, werden diese durch
den Generator dementsprechend in dem \textit{AppEntryPoint.java} positioniert.
Alle anderen vorhandenen permanenten Views sollen dazwischen eingetragen werden
und müssen später durch den Entwickler positioniert werden, da dies
layoutspezifisch ist.

Der Entwickler soll weiterhin zusätzliche Änderungen vornehmen müssen bzgl. der Startseite der GWT
Webanwendung, welches in dem generiertem Code anstatt
konkreter Angabe des Klassennamens innerhalb des Befehls über
\grqq{}Start\grqq{} gekennzeichnet wird. Dieses Vorgehen kann das Verständnis
für die Architektur stärken und es werden überflüssige und überfüllende
Attribute innerhalb des Profils und M1-Modells vermieden.

allg. zum Generator z.B. Strukturierung, Redundanz usw.
\section{Anwendungsfall}
\label{Anwendungsfall}
Prinzipiell soll es über den Generator möglich sein verschiedene Anwendungsfälle
zu generieren. Beispielhaft soll dementsprechend eine
einfache Homepage mit u.A.
einem Portfolio und einer Newsseite sowie einem Header über dem alle Seiten
erreichbar sind als Anwendungsfall dienen. Weiterhin soll mit 2
M1-Modellen gearbeitet werden, in dem der Anwendungsfall inhaltlich leicht abgeändert wird.
Ein M1-Modell dient Testzwecken und das andere der Vervollständigung der
gesamten Homepage mit Änderungen in dem generiertem Code z.B. zur Gestaltung der
Homepage und einer vollständigen und geordneten Projektstruktur.
Dadurch können alle Test- und verschiedene Anwendungsfälle abgedeckt werden,
inklusive der Einbindung von View Komponenten, und eine Trennung zwischen notwendigen Änderungen im
generiertem Code, durch ein einheitliches Projekt, erfolgen.