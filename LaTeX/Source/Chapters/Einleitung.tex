\chapter{Einleitung}
\label{Einleitung}
Dieses Projekt entstand im Rahmen des Software Engineering - Advanced Topic
Kurses im Master Studiengang Medieninformatik, der Beuth Hochschule Berlin, im
Wintersemester 2013/14.

Die Ausarbeitung beschäftigt sich mit der Generierung eines Google Web
Toolkit (GWT) Projektes.
Dies erfolgt durch eine Model-To-Text Transformation mittels eines Model
Transformation Language (MTL) Generators, welchem mit der Object Constraint Language (OCL), weitere
Funktionalitäten zur Verfügung stehen. 

Dabei wird ein besonderes Augenmerk auf die Architektur gelegt, allerdeings ohne
des Einsatzes von Layouting. Deshalb findet eine Generierung anhand einer
vorgegebenen Grundstruktur bzw. -architektur statt. Mit dem in der vorliegenden
Projektarbeit entwickelten Generator, ist es einem Entwickler möglich eine
komplette GWT Struktur generieren zu lassen. Auf Grundlage eines, auf einem UML
Profil basierenden, Klassendiagramms kann eine komplette GWT Frontend
Webanwendung abgebildet und mit kleinen Änderungen schnell lauffähig gemacht werden.
Das UML Profil wird so gestaltet, dass eine GWT Anwendung gemäß der vorgegebenen
Architektur einfach umgesetzt werden kann. Dabei finden das
Model-View-Presenter (MVP) Pattern sowie verschiedene Frameworks, welche
weiterhin Architekturkonzepte, beispielsweise für Dependency Injection, liefern,
als grundlegende Konstrukte Verwendung. Anstoß zu diesem Projekt war eine
gegebene \glqq{}Best Practice\grqq{} Lösung mehrerer Entwickler, die jedoch
ziemlich aufwändig und fehleranfällig in der Umsetzung ist.
\\\\
In dem weiterem Verlauf dieser Ausarbeitung werden die zugrundeliegenden
Konzepte und Frameworks (vgl. Abschnitt \ref{Grundlagen}) vorgestellt. Weiterhin
wird die Idee und konzeptionellen Aspekte (vgl. Abschnitt \ref{Konzeption}),
auch hinsichtlich der Ziel-Architektur, erläutert. Diese werden im
Weiterem mit sich ergebenden Änderungen umgesetzt (vgl. Abschnitte \ref{UMLProfil},
\ref{M1Modell}, \ref{Generator}).
Anhand des Ergebnisses (vgl. Abschnitt \ref{Ergebnis}) wird gezeigt, dass eine
GWT Anwendung anhand zweier Anwendungsfälle generiert und wie diese lauffähig
gemacht werden kann. Zusammenfassend folgt eine Bewertung des Generator
Projektes mit einem Ausblick auf Erweiterungsmöglichkeiten (vgl. Abschnitt
\ref{FazitAusblick}).
