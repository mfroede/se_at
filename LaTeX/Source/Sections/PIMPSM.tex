\subsection*{Platform Independent Model und Plattform Specific Model} \label{PIMPSM}
Das Platform Independent Model (PIM, dt. plattformunabhängiges Modell) stellt ein
Softwaresystem dar, das unabhängig von der technologischen Plattform ist. Zudem wird die konkrete technische Umsetzung des Systems nicht berücksichtigt. In dem PIM
sind alle Anforderungen erfasst. Alles, was es im System zu spezifizieren gibt, ist definiert,
jedoch komplett frei von der später folgenden Implementierung. Somit ist nicht nur eine
einzige Implementierung des Systems möglich, sondern durchaus mehrere
unterschiedliche.
Werden nun die Funktionalitäten kombiniert, die im Platform Independent Model definiert sind, mit den Designanforderungen der gewünschten Plattform, so entsteht das Platform Specific Model (PSM, dt. plattformspezifisches Model). Dies geschieht über Modelltransformationen. Das nun entstandene PSM kann durch weitere Transformationen immer spezifischere Modelle erstellen, bis letztendlich der Quellcode für eine Plattform generiert wird. Im Gegensatz zum PIM, welches nur die
fachlichen Anforderungen definiert, werden beim PSM auch die technischen Aspekte
eingebunden.\cite[S.377 ff.]{bib:MDA2}\cite{bib:MDA3}\\ 
 
Allerdings ist zu beachten, dass es sich bei PIM und PSM um relative Konzepte handelt. 
In diesem Projekt sind sowohl M1 als auch M2 Modell im Bezug auf GWT als PIM zu betrachten. Die eigentliche Spezifizierung für GWT erfolgt erst bei der Model-To-Text Transformation im Generator.