\subsection*{UiBinder}
\label{UiBinder}
Das UiBinder Framework für GWT Anwendungen ist ähnlich zu betrachten wie HTML
und CSS. Dieses Framework ermöglicht das Layouting von GWT Webseiten. Dabei
wird die Webseite nicht länger ausschließlich in Java Code gestaltet, sondern
deren View Komponenten und Layout über eine spezielle, HTML ähnliche, Expression
Language deklarativ formuliert. Dies geschieht innerhalb einer
\texttt{.ui.xml}-Datei. In dieser Sprache können neben View Komponenten und
Layout auch Styles innerhalb eines \texttt{ui:Style}-Tags (vgl. Listing
\ref{lst:BSPCodeUIXML}) definiert werden. Dies bietet die Möglichkeit die View
Implementierung zu entkoppeln. Darüber hinaus existieren noch weitere
Möglichkeiten zur Entkopplung der View Implementierung, sodass beispielsweise
statische View Komponenten nur innerhalb der \texttt{.ui.xml}-Datei enthalten
sind und somit ein Überladen der View Implementierung vermindert werden kann.
Weiterhin können die View Komponenten innerhalb dieser Datei an die
View Komponenten in der View Implementierung gebunden werden\cite{bib:uiBind}.
Dazu folgende Codeauszüge von einem vorangegangenem GWT Projekt zur Erläuterung
dieses Zusammenhangs.\\
\lstset{language=gwt}
\begin{lstlisting}[caption={Beispielcode UiBinder in View
Implementierung}, label={lst:BSPCodeView}]
	private static LoginViewImplUiBinder $uiBinder$ = GWT
			.create(LoginViewImplUiBinder.class);

	interface LoginViewImplUiBinder extends 
			UiBinder<Widget, LoginViewImpl> {}

	|@UiField|
	TextBox $name$;
		
	//Constructor
	|@Inject|
	public LoginViewImpl() {
		$content$.add($uiBinder$.createAndBindUi(this));
	}
	|@UiHandler|({ "button" })
	void onButtonPressed(ClickEvent e) {
		// do something
	}
\end{lstlisting}
\lstset{language=uixml}
\begin{lstlisting}[caption={Beispielcode UiBinder in \texttt{.ui.xml}},
label={lst:BSPCodeUIXML}]
<!DOCTYPE ui:UiBinder SYSTEM 
	"http://dl.google.com/gwt/DTD/xhtml.ent">
<ui:UiBinder xmlns:ui="urn:ui:com.google.gwt.uibinder"
	xmlns:g="urn:import:com.google.gwt.user.client.ui"
	xmlns:my="urn:import:myprojectpackage">
	<ui:style>
		.enterbutton {
			font-size: 16px;
			font-weight: bold;
			padding: 10px;
			color: #336699;
		}
	</ui:style>
	<g:FlowPanel>
		<g:Label text="Anmeldename"></g:Label>
		<g:TextBox ui:field="name"></g:TextBox>
		<g:Button text="Einloggen" ui:field="button" 
			styleName="{style.enterbutton}"></g:Button>
	</g:FlowPanel> 
</ui:UiBinder> 
\end{lstlisting}
Die Annotationen \texttt{@UiField} und \texttt{@UiHandler} in der View
Implementierung (vgl.
Listing \ref{lst:BSPCodeView}) ermöglichen den Zugriff auf die View Komponenten
mit dem jeweiligem Attribut \texttt{ui:field} in der \texttt{.ui.xml}-Datei
(vgl.
Listing \ref{lst:BSPCodeUIXML}). Mit \texttt{@UiField} ist es möglich die
dafür zuständige Instanz, der im UiBinder-Template definierten View
Komponente, zu erhalten.
Diese kann dann über den Java Code instanziiert oder Style Eigenschaften gesetzt werden.
Die Annotation \texttt{@UiHandler} dagegen ermöglicht die Anmeldung einer
Methode auf der Instanz. Darüber kann dann im Falle des Beispiels ein Klick Event auf dem Button
ausgeführt werden.

Damit zeigen die Listings \ref{lst:BSPCodeView} und \ref{lst:BSPCodeUIXML}
nur kleine Beispiele für die Nutzung des UiBinder Frameworks, welche
innerhalb des Generator Projektes umgesetzt werden sollen.