\subsection{UiBinder}
\label{UiBinder}
Das UiBinder Framework für GWT Anwendungen ist ähnlich zu betrachten wie HTML
und CSS. Dieses Framework ermöglicht das Layouting von GWT Websites. Dabei
wird die Website nicht nur über den Code erzeugt, sondern zusätzlich mittels
einer xml-Datei, die ui.xml Datei. In dieser Datei können z.B. Style
Eigenschaften ähnlich wie bei CSS gesetzt werden innerhalb eines ui:Style
Tags (vgl. Listing \ref{lst:BSPCodeUIXML}). Dies bietet die Möglichkeit die
View Implementierung zu entkoppeln. Darüber hinaus existieren noch weitere
Möglichkeiten zur Entkopplung der View Implementierung z.B.
sodass die View Komponenten die statisch sind z.B. nur innerhalb der ui.xml
Datei enthalten sind und somit ein Overload der View Implementierung vermindert
werden kann. Weiterhin können die Komponenten innerhalb dieser Datei gebunden
werden an die, in der View Implementierung enthaltenen Komponenten
\cite{bib:uiBind}. Dazu folgender Codeauszug von einem vorangegangenem GWT
Projekt zur Erläuterung dieses Zusammenhangs.\\
\lstset{language=gwt}
\begin{lstlisting}[caption={Beispielcode UiBinder in View
Implementierung}, label={lst:BSPCodeView}]
	private static LoginViewImplUiBinder $uiBinder$ = GWT
			.create(LoginViewImplUiBinder.class);

	interface LoginViewImplUiBinder extends 
			UiBinder<Widget, LoginViewImpl> {}

	|@UiField|
	TextBox $name$;
		
	//Constructor
	|@Inject|
	public LoginViewImpl() {
		$content$.add($uiBinder$.createAndBindUi(this));
	}
	|@UiHandler|({ "button" })
	void onButtonPressed(ClickEvent e) {
		// do something
	}
\end{lstlisting}
\lstset{language=uixml}
\begin{lstlisting}[caption={Beispielcode UiBinder in ui.xml},
label={lst:BSPCodeUIXML}]
<!DOCTYPE ui:UiBinder SYSTEM 
	"http://dl.google.com/gwt/DTD/xhtml.ent">
<ui:UiBinder xmlns:ui="urn:ui:com.google.gwt.uibinder"
	xmlns:g="urn:import:com.google.gwt.user.client.ui"
	xmlns:my="urn:import:myprojectpackage">
	<ui:style>
		.enterbutton {
			font-size: 16px;
			font-weight: bold;
			padding: 10px;
			color: #336699;
		}
	</ui:style>
	<g:FlowPanel>
		<g:Label text="Anmeldename"></g:Label>
		<g:TextBox ui:field="name"></g:TextBox>
		<g:Button text="Einloggen" ui:field="button" 
			styleName="{style.enterbutton}"></g:Button>
	</g:FlowPanel> 
</ui:UiBinder> 
\end{lstlisting}
Die Annotationen @UiField und @UiHandler in der View Implementierung (vgl.
Listing \ref{lst:BSPCodeView}) ermöglichen den Zugriff auf die View Komponenten
mit dem jeweiligem Attribut ui:field in der ui.xml (vgl. Listing
\ref{lst:BSPCodeUIXML}). @UiField ist dabei dafür zuständig die Instanz zu
erhalten. Diese kann dann z.B. über den Java Code definiert oder Style
Eigenschaften gesetzt werden. Entgegen dem ermöglicht @UiHandler die Anmeldung
einer Methode auf der Instanz. Darüber kann dann im Falle des Beispiels ein
Klick Event auf dem Button ausgeführt werden.

Damit zeigen die Listings \ref{lst:BSPCodeView} und \ref{lst:BSPCodeUIXML}
nur kleine Beispiele für die Nutzung von dem UiBinder Framework, welche
innerhalb des Generator Projektes umgesetzt werden sollen.



