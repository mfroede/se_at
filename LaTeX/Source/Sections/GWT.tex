\section{GWT}
\label{GWT}
Das Google Web Toolkit, kurz GWT, ist ein open-source Projekt von Google. GWT
ist ein Framework, welches genutzt werden kann um Webanwendungen mittels der
Programmiersprache Java zu implementieren. Dabei übersetzt GWT den gesamten
Source-Code in JavaScript Code und DOM-Elemente. Während der Übersetzung des
Java Codes zu JavaScript Code werden darüber hinaus Optimierungen vorgenommen
wie das Löschen von dead-Code. Dies führt potenziell dazu, dass komplexe
Anwendungen im Browser schneller ausgeführt werden können. Darüber hinaus bietet
GWT noch weitere Möglichkeiten, die dem Entwickler einer Webanwendung zu Gute
kommen. Dazu zählen u. A. das Integrieren von JavaScript Code oder von
JavaScript Bibliotheken innerhalb des Java Codes durch das JavaScript Native
Interface, kurz JSNI und das sogenannte Code-Splitting, welches einem Entwickler
ermöglicht sogenannte Split Points innerhalb des Codes zu setzen, welche dazu
führen, dass bei der Ausführung der Anwendung bestimmte Inhalte ab dem Split
Point später nachgeladen werden und dadurch die Startladezeit verringern.
Google bietet mit zu den genannten Eigenschaften weitere positive
Software Engineering Aspekte. Durch GIN (GWT INjection) 
