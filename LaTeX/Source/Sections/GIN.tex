\section{Dependency Injection mittels GIN} \label{GIN}
Bei Dependency Injection handelt es sich um einen Begriff aus der
objektorientierten Programmierung, welcher erstmalig von Martin Fowler 2004
verwendet wurde \cite{bib:DI}. Dabei handelt es sich um
ein Verfahren, bei dem zur Laufzeit eines Programmes zusätzliche Informationen,
beispielsweise beim Aufruf einer Funktion, zur Verfügung gestellt werden. Dies
wird von einem extra Dependency Injection Framework vorgenommen, in dieser
Arbeit handelt es sich dabei um das GIN Framework.
\\

GIN ist ein Framwork für Dependency Injection, es wurde von Google
für GWT entwickelt \cite[GIN]{bib:gin}. GIN setzt auf Google Guice
\cite[Guice]{bib:guice} auf und erweitert die Java Dependency Injection für den
speziellen Anwendungsfall von GWT. Die zu generierende Ziel-Architektur nutzt
GIN zur Dependency Injection, es wurde direkt in den GWT Generator eingebaut,
wodurch Dependency Injection teilweise schon zur Compile Zeit erfolgen kann,
dies sorgt dafür, dass es kaum Laufzeit Overhead gibt.

Guice wurde 2008 von Google für Dependency Injection mit Java entwickelt und war
das erste Framwork, das dieses mit Hilfe von Annotationen
ermöglicht hat. Bei Guice werden mittels \texttt{bind}-Befehlen eine Verbindung
zwischen Interfaces und deren konkreten Klassen hergestellt, dadurch sind die konkreten
Implementierungen leichter austauschbar.
