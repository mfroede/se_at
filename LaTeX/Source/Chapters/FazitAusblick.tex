\chapter{Fazit und Ausblick}
\label{FazitAusblick}
Anhand des Ergebnisses wird deutlich, dass mit dem Generator Projekt und einigen
kleineren Änderungen im generiertem Code eine funktionierende GWT Frontend
Anwendung erzeugt werden kann. Durch einen
hohen Abstraktionsgrad innerhalb des UML-Profils wird die Entwicklung einer solchen Anwendung 
vereinfacht und die Fehleranfälligkeit auf ein geringes Maß minimiert ohne
Einschränkungen bei der vorgegebenen Zielarchitektur (vgl. Abschnitt
\ref{AufBZielArchitektur}).
Darüber hinaus sind die durch GWT und MVP gegebenen Vorteile z.B. der
einfache Austausch von Views weiterhin und teilweise leichter nutzbar z.B.
durch Aus- und Einkommentierung bei dem Austausch von Views. Zur Gestaltung der
Website stehen einem Entwickler alle Möglichkeiten z.B. die
Nutzung von Ui-Editoren weiterhin ohne Einschränkungen zur Verfügung und die
Anwnendungsfälle sind frei wählbar, wie anhand der 2 M1-Modelle deutlich wird.
Es können eigene View Elemente erstellt und eingebunden werden, welches die
Architekturkonzepte zusätzlich unterstützt. Des weiteren wird die Navigation
über ViewNavigationObjects innerhalb der Seiten generiert, jedoch ist weiteres
Verhalten wie z.B. das Öffnen eines Popups nicht umgesetzt.

Durch Abschnitt \ldots ref wird ersichtlich, dass die vorzunehmende Änderung im
Bereich der View Komponenten innerhalb der ui.xml Datei keine optimale Lösung ist. Dies wird hervorgerufen
dadurch, dass ViewObjects weitere ViewObjects beinhalten können und somit in der
ui.xml mehrfach enthalten sein können. Dies führt zu dem Gedanken, dass eine
andere Generierungssprache außerhalb von OCL potenziell besser geeignet wäre, da
keine weitere Möglichkeit gefunden werden konnte einen optimalen Lösungsansatz
umzusetzen u.A. durch das verändern einer Variable innerhalb einer if-Bedingung.
Eine weitere Möglichkeit der Optimierung an dieser Stelle könnte darin bestehen,
dass Backend generierbar zu machen. Dadurch können Datenmodelle zu dem
UML-Profil hinzugefügt werden, welche auch die Grundlage für View Komponenten
bilden können. Anhand eines
Beispiels kann das Datenmodell einer Produktklasse mit den Attributen Name,
Preis und Verkaufsort dazu genutzt werden, dass innerhalb einer View automatisch
eine Tabelle generiert wird, welche als Spalten die Attribute beinhaltet und
alle Produkte anzeigt. Dieser Lösungsansatz wäre einerseits eine
Weiterentwicklung des Generatorprojekten bietet jedoch noch keine Komplettlösung, sollten Views auch
ohne Datenmodelle generiebar sein.

Weiterhin müssen strukturelle Änderungen z.B. das Umsetzen der
index.html oder das Hinzufügen von Bibliotheken z.B. GIN (vgl. Abschnitt ref
\ldots) innerhalb des GWT Projektes vorgenommen werden. Diese Änderungen können
durch die Verbindung des Generator Projektes mit Maven vermieden werden.

Das zu generierende M1-Modell weist keine Assoziationen auf, wodurch viele
Eigenschaften wie das Beinhalten von eigenen View Objekten in Views versteckt
bleiben. Aus diesem Grund wäre eine Generierung eines UML Klassendiagramms als
Weiterentwicklung ein wichtiger aspekt. Dadurch wird zusätzlich der im Fokus
liegende architektonische Aspekt des Generator Projektes hervorhebt. Darüber
hinaus sind viele Teile wie z.B. die einmalig vorhandenen Klassen sowie die View
Klassen z.B. Activity, Place und ViewImpl.ui.xml in dem M1-Modell nicht
ersichtlich bzw. nicht vorhanden, wodurch das generierte unübersichtlich werden
kann. Weshalb ein UML Klassendiagramm weiterhin stark unterstützend wirkt. 

Eine weitere Möglichkeit Übersicht zu schaffen besteht zusätzlich darin
Activitätsdiagramme zu generieren. Dies ermöglicht einen weiteren Überblick über
die generierte Navigation.

Zusammenfassend ist zu erwähnen, dass das Generatorprojekt eine gute
Grundlage für die Weiterentwicklung darstellt unter der Vorrausetzung, dass die
Redundanz und Strukturierung des Generators verbessert wird.