\chapter{Fazit und Ausblick}
\label{FazitAusblick}
Anhand des Ergebnisses wird deutlich, dass mit dem Generator Projekt und einigen
kleineren Änderungen im generiertem Code eine funktionierende GWT Frontend
Anwendung erzeugt werden kann. Durch einen
hohen Abstraktionsgrad innerhalb des UML Profils wird die Entwicklung einer
solchen Anwendung vereinfacht und die Fehleranfälligkeit auf ein geringes Maß minimiert ohne
Einschränkungen bei der vorgegebenen Ziel-Architektur (vgl. Abschnitt
\ref{AufBZielArchitektur}).
Darüber hinaus sind die durch GWT und MVP gegebenen Vorteile z.B. der
einfache Austausch von Views weiterhin und teilweise leichter nutzbar
beispielsweise durch Aus- und Einkommentierung bei dem Austausch von View
Implementierungen.
Zur Gestaltung der Webseite stehen einem Entwickler alle Möglichkeiten z.B. die
Nutzung von UI Editoren weiterhin ohne Einschränkungen zur Verfügung und die
Anwendungsfälle sind frei wählbar, wie anhand der zwei M1 Modelle deutlich wird.
Es können eigene View Elemente in Form von \textit{OwnViewObjects} erstellt und
eingebunden werden, welches die Architekturkonzepte zusätzlich unterstützt. Des Weiteren wird die Navigation
über \textit{ViewNavigationObjects} innerhalb der View Implementierung
generiert, jedoch sind weitere Verhalten wie das Öffnen eines Popups nicht
umgesetzt. Zusätzlich sind nicht alle View sowie View Komponenten abhängige
imports, da nur notwendige imports generiert werden sollen, vollständig umgesetzt
und müssen somit vervollständigt werden.

Durch Abschnitt \ref{Probleme} wird ersichtlich, dass die
vorzunehmende Änderung im Bereich der View Komponenten innerhalb der \texttt{.ui.xml}-Datei keine optimale
Lösung ist. Dies wird hervorgerufen dadurch, dass \textit{ViewObjects} weitere
\textit{ViewObjects} beinhalten können und somit in der \texttt{.ui.xml}
mehrfach enthalten sein können. Dies führt zu dem Gedanken, dass eine andere
Generierungssprache außerhalb von OCL potenziell besser geeignet wäre, da keine
weitere Möglichkeit gefunden werden konnte einen optimalen Lösungsansatz
umzusetzen u.A. durch das Verändern einer Variable innerhalb einer
\texttt{if}-Bedingung (vgl. Abschnitt \ref{Probleme}).
Eine weitere Möglichkeit der Optimierung an dieser Stelle könnte darin bestehen,
Datenmodelle generierbar zu machen und zu dem UML Profil hinzuzufügen. Diese
können als Grundlage für View Komponenten dienen.
Anhand eines Beispiels kann das Datenmodell einer Produktklasse mit den Attributen Name,
Preis und Verkaufsort dazu genutzt werden, dass innerhalb einer
dementsprechenden View Implementierung automatisch eine Tabelle generiert wird,
welche als Spalten die Attribute beinhaltet und alle Produkte anzeigt. Dieser Lösungsansatz wäre einerseits eine
Weiterentwicklung des Generator Projektes, bietet jedoch noch keine
Komplettlösung, sollten View Komponenten auch ohne Datenmodelle
generierbar sein.

Weiterhin wäre es denkbar ein Backend genierbar zu machen und die
genannten Datenmodelle zusätzlich darauf zugreifen zu lassen. Damit würden
diese das Model im MVP bilden und eine weitere architektonisch sinnvolle Abgrenzung zwischen Frontend
und Backend stattfinden.

Innerhalb des Generators wurde bereits in Abschnitt \ref{Probleme} ersichtlich,
dass nicht alle Optimierungen im Bereich Struktur und Redundanz vorgenommen wurden.
Dies muss weiterhin geschehen, unterliegt jedoch trotzdem dem Aspekt der
Verwendbarkeit von OCL. Dies liegt daran, dass viele
unterschiedliche \texttt{if}-Bedingungen und viele \texttt{if}-Bedingungen mit
ähnlichen \texttt{if-else}-Zweigen enthalten sind. Eine Überlegung wie dies mit
OCL lösbar wäre ist weiterhin erforderlich, wobei ein Test mit anderen Generierungssprachen,
die diesbezüglich mehr Möglichkeiten zur Strukturierung bieten, denkbar ist.

Weiterhin müssen strukturelle Änderungen z.B. das Umsetzen der
\texttt{index.html} oder das Hinzufügen von Bibliotheken z.B. GIN (vgl.
Abschnitt \ref{Ergebnis}) innerhalb des GWT Projektes vorgenommen werden. Diese
Änderungen können durch die Verbindung des Generator Projektes mit Maven vermieden werden.
\\\\
Das zu generierende M1 Modell weist keine Assoziationen auf, wodurch viele
Eigenschaften wie das Beinhalten von eigenen View Objekten in View
Implementierungen versteckt bleiben. Aus diesem Grund wäre eine Generierung
eines UML Klassendiagramms als Weiterentwicklung ein wichtiger Aspekt. Dadurch wird zusätzlich der im Fokus
liegende architektonische Aspekt des Generator Projektes hervorgehoben.
Darüber hinaus sind viele Teile wie z.B. die einmalig vorhandenen Klassen sowie die View
Klassen \texttt{Activity}, \texttt{Place} und \texttt{ViewImpl.ui.xml} in
dem M1 Modell nicht ersichtlich bzw. nicht vorhanden, wodurch das generierte
unübersichtlich werden kann. Weshalb ein UML Klassendiagramm weiterhin stark unterstützend wirkt. 

Eine weitere Möglichkeit Übersicht zu schaffen besteht zusätzlich darin
ActivityDiagramme zu generieren. Dies ermöglicht einen weiteren Überblick über
die generierte Navigation.

Zusammenfassend ist zu erwähnen, dass das Generator Projekt eine gute
Grundlage für die Weiterentwicklung darstellt, unter der Vorrausetzung, dass die
Redundanz und Strukturierung des Generators verbessert wird.