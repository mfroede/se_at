\chapter{Ergebnis} \label{Ergebnis}
Das Ziel bestand darin eine Grundarchitektur für ein GWT Projekt zu erzeugen,
welche es dem Entwickler vereinfachen soll seine Anwendung gemäß dieser
Architektur zu erstellen. Um das Projekt vollständig lauffähig zu machen, sind
noch ein paar Handgriffe vorzunehmen.

\begin{itemize}
  \item \texttt{AppEntryPoint.java} \\
  Hier muss definiert werden, welche der Views die Startseite werden soll.
  \item \texttt{AbstractView.java} \\
  In dieser Klasse ist es möglich Standardeigenschaften für alle View
  Implementierungen zu bestimmen, der Befehl für die Größe ist hier
  vorgeneriert und muss vervollständigt werden.
  \item \texttt{ProductionGinModule.java} \\
  Auch hier muss wie in der \texttt{AppEntryPoint}-Klasse noch einmal die
  Starteite definiert werden.
  \item \texttt{index.html und style.css} \\
  Diese beiden Dateien müssen nach dem Durchlauf des Generators in den
  \texttt{war}-Ordner verschoben werden. Sie dienen als Ausgangspunkt für die
  Webanwendung.
  \item \texttt{Zusätzliche Bibliotheken}\\
  	Zusätzlich zu dem \texttt{GWT SDK} und \texttt{App Engine SDK} müssen die
    hier aufgeführten \texttt{.jar} Bibliotheken in das Projekt hinzugefügt
    werden.
  	\begin{itemize}
  	  \item \texttt{gin-2.0.jar}
  	  \item \texttt{guice-3.0-no\_aop.jar}
  	  \item \texttt{guice-assistedinject-3.0.jar}
  	  \item \texttt{gwt-servlet.jar}
  	  \item \texttt{javax.inject.jar}
	\end{itemize}
\end{itemize}
Derzeit sind noch folgende Schritte notwendig.
\begin{itemize}
  \item \texttt{import} \\
  Es ist notwendig die noch fehlenden Imports in den Java Klassen
  einzufügen, da diese aus Zeitgründen nicht vollständig eingearbeitet
  worden sind, um nicht überflüssige Imports zu generieren. 
  \item \texttt{'Name'ViewImpl.ui.xml} \\
  In diesen Dateien müssen doppelte Elemente entfernt werden, die der Generator zu
  viel eingefügt hat, näheres dazu in Abschnitt \ref{FazitAusblick} zweiter
  Absatz.
\end{itemize} 

Sind diese Maßnahmen durchgeführt, ist das Projekt lauffähig. Zusätzlich können
weitere Änderungen vorgenommen werden, zum Beispiel können weitere Elemente
eingefügt oder das Design angepasst werden, dieses wurde von dem Generator nur
rudimentär, in Form einer \texttt{css}-Datei angelegt.

\section{Beispiel Anwendung}
Für das in diesem Abschnitt gezeigte Beispiel wurde das M1 Modell aus Abbildung
\ref{Fig:ergModell} verwendet. Zu sehen sind fünf Seiten, welche in Pakete
gegliedert sind. Darüber hinaus wurden zwei \textit{PermanentViewImpl}
modelliert. Eine betrifft den Header, mit einem eingebautem Menu und eine den
Footer, in dem eine Navigation zum Impressum vorgesehen ist.

\begin{figure}[htbp]
\begin{center}
\includegraphics[width=0.8\textwidth]{./img/Model2.png}
\caption{Verwendetes Modell}\label{Fig:ergModell}
\end{center}
\end{figure}

Durch die extra eingebauten Pakete erweitert sich die Grundstruktur des
Projektes. In Abbildung \ref{Fig:packegeModel} ist auf der linken Seite die
komplette Paketstruktur des Projektes zu sehen. Hier ist zu erkennen, dass
für die fünf Seiten separate Pakete existieren, welche in Abbildung
\ref{Fig:ergModell} zusehen sind. Jedoch für den Header und den Footer keine
zusätzlichen Pakete angelegt worden sind. Auf der rechten Seite in Abbildung
\ref{Fig:packegeModel}, wird der Inhalt einiger der Pakete näher gezeigt. Hier
ist zu erkennen, dass die beiden \texttt{ViewImpls}, die nicht in Paketen
liegen, direkt in dem \texttt{View}-Paket liegen. Zusätzlich sind die Klassen
\texttt{Activity}, \texttt{Place}, \texttt{View} und die \texttt{.ui.xml}-Datei
in dem Paket zu sehen. Am Beispiel der \texttt{Home}-Seite ist gezeigt, dass die
hierfür generierten Klassen und Dateien in einem separaten Paket von
\texttt{view}, nämlich in \texttt{home} liegen.

\begin{figure}[htbp]
\begin{center}
\raisebox{3.2cm}{\includegraphics[width=0.4\textwidth]{./img/Packages.png}}
\includegraphics[width=0.4\textwidth]{./img/PackagesExample.png}
\caption{Generierte Paketstruktur, links gesamte
Paketstruktur im Überblick; rechts Beispiel für das
Strukturieren im Modell, mit zusätzlichen Paketen}\label{Fig:packegeModel}
\end{center}
\end{figure}
 
In diesem Beispiel soll die \texttt{Home}-Seite die Startseite für die
Webanwendung darstellen. Aus diesem Grund wurde entsprechend in der
\texttt{AppEntryPoint}- und der \texttt{ProductionGinModule}-Klasse die
jeweilige Klasse für die StartView angegeben (vgl. Listing
\ref{lst:appEntryPoint} und \ref{lst:ProductionGinModule}).
\lstset{language=gwt}
\begin{lstlisting}[caption={Änderung an der \texttt{AppEntryPoint}-Klasse zur
Bestimmung der Startseite}, label={lst:appEntryPoint}] 
[..]
  $historyHandler$.register($injector$.getPlaceController(), 
      $eventBus$, new HomePlace());
[..]
\end{lstlisting}
\lstset{language=gwt}
\begin{lstlisting}[caption={Änderung an der \texttt{ProductionGinModule}-Klasse
zur Bestimmung der Startseite}, label={lst:ProductionGinModule}] 
[..]
  bind(HomeView.class);
[..]
\end{lstlisting}

Zusätzlich wurde an einigen Stellen der Inhalt der Seiten erweitert. Dies ist
einerseits dadurch möglich, dass innerhalb der \texttt{ViewImpl}-Klasse
zusätzlicher Inhalt eingefügt werden kann (vgl. Listing \ref{lst:conntentJava})
und andererseits auch innerhalb der \texttt{.ui.xml}-Datei (vgl. Listing
\ref{lst:conntentUI}).

\lstset{language=gwt}
\begin{lstlisting}[caption={Einfügen von Inhalten auf einer Seite durch
Veränderungen am Java-Code},
label={lst:conntentJava}] 
[..] 
  HTML $html$ = new HTML("<div><h1>Brandhei&szlig;e News</h1>"
	+"Lorem ipsum [..] amet."
	+"<div/>");
  $content$.add($html$);
[..]
\end{lstlisting}
\lstset{language=uixml}
\begin{lstlisting}[caption={Einfügen von Inhalten auf einer Seite durch
Veränderungen in der \texttt{.ui.xml}-Datei},
label={lst:conntentUI}] 
[..] 
 <g:FlowPanel>
	<!-- InteractionElements -->
	<!-- Start of user code Home 
	     Start protectetRegion -->
			<g:Label text=" Lorem [..] facilisi. "/>
			[..]		
	<!-- End of user code -->
	</g:FlowPanel>
[..]
\end{lstlisting}

Dieses sind nur zwei Arten wie die Anwendung weiter bearbeitet werden kann,
nachdem der Quellecode aus dem M1 Modell generiert worden ist. Zum Einen
unterstützen der Generator und das Profil derzeit nicht alle von GWT
bereitgestellten UI Elemente, sodass diese nachträglich eingefügt werden
müssen. Da diverse UI Editoren existieren, welche das Erzeugen einer
Benutzeroberfläche grafisch unterstützen, wurde sich in dieser Ausarbeitung
bewusst gegen das Erzeugen einer Webanwendung, die vollständig designt wurde,
entschieden. 
