\section{GWT}
\label{GWT}
Das Google Web Toolkit, kurz GWT, ist ein open-source Projekt von Google. Es
dient der Entwicklung von Webanwendungen mittels Java. Dabei übersetzt der GWT
Compiler den gesamten Java Source-Code in JavaScript Code.
Zudem werden während der Übersetzung Code Optimierungen vorgenommen, welche u.A.
das Löschen von nicht benötigtem Code, z.B beim Einsatz von mehreren
Browsern als Plattformen, betreffen. 

Weiterhin bietet GWT die Möglichkeit zur Interaktion mit JavaScript durch
das JavaScript Native Interface, kurz JSNI.
Dadurch können JavaScript Bibliotheken angebunden bzw. verwendet oder die
Nutzung von JSON erleichtert werden \cite[S. 4-9]{bib:GWTinAction}\cite[S.
237-238]{bib:GWToReilly}.

Darüber hinaus ist dadurch eine Kommunikation mit dem Backend
möglich. Zusätzlich kann diese Kommunikation durch Remote Procedure Calls,
kurz RCPs, u.A. mittels RequestBuilder oder GWT-RCP erfolgen. Beide setzen auf
dem XMLHttpRequest JavaScript Objekt auf, welches die Kommunikation zwischen dem
Browser und dem Server ohne Seitenneuladen erlaubt. Der RequestBuilder ist ein
Wrapper für das genannte JavaScript Objekt und GWT-RCP ermöglicht den Austausch
von konkreten Java Objekten \cite[S. 16]{bib:GWTinAction}\cite[S.
222]{bib:GWToReilly}.

Dies zeigt einen kurzen Einblick in die verschieden Möglichkeiten mit
GWT, welche folgend nicht näher erläutert werden, weil eine Generierung einer
GWT Frontend Anwendung ohne Server Kommunikation erfolgen soll.

Der Einsatz mit GWT ist flexibel und durch den GWT Compiler wird 
eine bessere Laufzeitausfürhung der Webanwendung erlangt\cite{bib:GWTStarted}.
Dies sind 2 Vorteile des Nutzens von GWT. Zusätzlich sind die durch Google
gebotenen Architekturkonzepte durch u.A. Model-View-Presenter, kurz MVP, ein
Ansatz und Grund einen Generator für GWT Frontend Anwnedungen zu schreiben.
Dafür werden im weiterem Verlauf MVP, UiBinder sowie GIN erläutert, welche die
Grundlage der umzusetzenden Architektur bilden.
