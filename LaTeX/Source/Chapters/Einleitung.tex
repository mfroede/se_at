\chapter{Einleitung}
\label{Einleitung}
Diese Ausarbeitung beschäftigt sich mit der Generierung eines Google Web
Toolkit (GWT) Projektes.
Dies erfolgt durch eine Model-To-Text Transformation mittels eines Model
Transformation Language (MTL) Generators, welchem mit der Object Constraint Language (OCL), weitere
Funktionalitäten zur Verfügung stehen. 

Dabei wird ein besonderes Augenmerk auf die Architektur gelegt, ohne den
Einsatz von Layouting. Deshalb findet eine Generierung anhand einer vorgegebenen
Grundstruktur bzw. -architektur statt. Mit dem erstelltem Generator Projekt in
dieser Ausarbeitung, ist es einem Entwickler möglich basierend auf einem M1 Modell in Form eines UML
Klassendiagramms eine vollständige Struktur einer GWT Frontend Webanwendung zu
erzeugen, welche mit kleineren Änderungen vollständig lauffähig ist. Die
Grundlage des M1 Modells bildet ein UML Profil, welches so gestaltet wird,
sodass eine GWT Frontend Webanwendung gemäß der vorgegebenen Architektur einfach
umgesetzt werden kann. Dies hat den Grund, dass die vorgegebene Architektur eine
durch den Einsatz des Model-View-Presenter (MVP) Patterns sowie verschiedener
Frameworks, eine durch verschiedene Entwickler angefertigte \glqq{}Best
Practice\grqq{} Lösung ist. Diese Herangehensweise ist jedoch aufwändig und
fehleranfällig in der Umsetzung. 
\\\\
In dem weiterem Verlauf dieser Ausarbeitung werden die zugrundeliegenden
Konzepte und Frameworks (vgl. Abschnitt \ref{Grundlagen}) vorgestellt. Weiterhin
wird die Idee und knozeptionellen Aspekte (vgl. Abschnitt \ref{Konzeption}),
auch hinsichtlich der Ziel-Architektur, erläutert, welche im Weiterem mit sich
ergebenden Änderungen umgesetzt (vgl. Abschnitte \ref{UMLProfil},
\ref{M1Modell}, \ref{Generator}) werden.
Anhand des Ergebnisses (vgl. Abschnitt \ref{Ergebnis}) wird gezeigt, dass eine
GWT Anwendung anhand zweier Anwendungsfälle generiert und mit kleinen
Änderungen lauffähig gemacht werden kann. Zusammenfassend folgt eine Bewertung
des Generator Projektes mit Erweiterungsmöglichkeiten (vgl. Abschnitt
\ref{FazitAusblick}).
