\section{Model Driven Architecture} \label{MDA}
Model Driven Architecture (dt. Modellgetriebene Architektur), kurz MDA, stellt
einen Ansatz zur Softwareentwicklung dar. Dieses Konzept ist 2001 von der
Object Management Group (OMG) veröffentlicht worden und gilt heute als
Standard. Hierbei werden Richtlinien zur Spezifikation in Form von Modellen vorgegeben.
Aus diesen Modellen, die formal eindeutig sind, wird dann mithilfe von Generatoren
automatisch der benötigte Code erzeugt. Ziel der MDA-Architektur ist es, den gesamten Prozess der Softwareerstellung in möglichst plattformunabhängigen Modellen darzustellen, sodass die Software zu einem hohen Anteil automatisch durch Transformationen von Modellen erzeugt werden kann. Die dabei entstehenden Transformatoren können eine hohe Wiederverwendbarkeit und Wartbarkeit sicherstellen.\cite[S. 79 f.]{bib:MDA1}\\
Bei den Modellen handelt es sich im Speziellen, um das Platform Independent Model und das Platform Specific Model, welche bei diesem Projekt auf das Metamodell der UML 2.4 Anwendung fanden.
Was dies genau bedeutet und wie die verschiedenen Modelle zu verstehen sind, wird in dem folgenden Abschnitt erläutert
