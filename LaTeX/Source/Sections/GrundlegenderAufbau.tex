% ---------------------------------------------
% ZielArchitektur
% ---------------------------------------------
\section{Ziel-Architektur}\label{AufBZielArchitektur}
Die für die Generierung vorgesehene Architektur basiert auf 
Architekturkonzepten verschiedener Entwickler und entstand bei der Entwicklung
von vorhergehenden GWT Projekten. Diese Architektur stellte sich dabei als
\glqq{}Best Practice\grqq{} Lösung heraus, welche jedoch aufwändig und
fehleranfällig bei der Umsetzung ist. Dies ist einer der Gründe für dieses Generator Projekt.
Im Folgendem wird die umzusetzende Architektur kategorisiert und anhand der zu
erstellenden Klassen und Dateien vorgestellt.
% -------------------Struktur Erklärungen--------------------------
\begin{itemize}
  \item einmalig vorhandene Dateien und Klassen
  \begin{itemize}
    \item \texttt{index.html}\\
    HTML Seite über die durch GWT, die in Java erzeugten Views
    eingebunden werden.
    \item \texttt{styles.css}\\
    CSS Datei für die Festlegung der Style Eigenschaften.
    \item \texttt{'Name'.gwt.xml}\\
    Konfigurationsdatei in der u.A. verwendete Bibliotheken sowie
    Browsereinstellungen und der \texttt{AppEntryPoint} eingetragen wird.
    \item \texttt{AppEntryPoint.java}\\
    Bildet die Einstiegsklasse für die GWT Anwendung. Darin sind elementare
    Strukturen festgelegt. Dazu zählt bespielsweise der grundlegende
    Webseiteninhalt, welcher zu dem \texttt{RootPanel} hinzugefügt und durch die
    \texttt{ViewActivityMapper} erreichbar ist, sowie die zum
    History Management gehörenden Komponenten, durch die die Startwebseite
    (durch Angabe des jeweiligen \texttt{Places}) registriert wird.
    \item \texttt{AbstractView.java}\\
    Die Oberklasse aller View Implementierungen innerhalb einer GWT
    Anwendung. Darin können grundlegende Eigenschaften wie die Größe aller
    View Implementierungen definiert werden.
    \item \texttt{AbstractActivityDefaultImpl.java}\\
    Diese Klasse ist ein GenericType mit dem Typparameter \texttt{Place} und
    wird von allen \texttt{Activity}-Klassen mit dem jeweiligem
    \texttt{Place} als Typparameter erweitert. 
 	Diese Klasse stellt eine Verbindung zwischen einem \texttt{Place} und der
 	dazugehörigen \texttt{Activity} her.
    \item \texttt{'Name'ViewActivityMapperImpl.java}\\
    Instanziiert die \texttt{Activity}-Klassen, außer die anderer
    \texttt{ViewActivityMapper}. Die Klasse existiert für jede vorhandene
    permanente View sowie prinzipiell einmal für alle normale
    Views. Durch die überschriebene Methode des GWT
    \texttt{ActivityMapper} \texttt{getActivity} wird mittels des \texttt{Place}
    die \texttt{Activity} zurückgeliefert. Dies wird benötigt, damit der
    Browserzugriff auf die jeweilige Webseite durch GWT umgesetzt werden kann.
    \item \texttt{AppPlaceHistoryMapper.java}\\
    Dient dem History Management, damit der Zugriff auf die View
    Implementierungen im Browser über den Place stattfinden und eine
    back-Funktionalität implementiert werden kann.
    \item \texttt{AppGinjector.java}\\
    Die Schnittstelle zum Zugriff u.A. auf die
    \texttt{ViewActivityMapper} sowie dem \texttt{EventBus}. Der
    \texttt{EventBus} dient wie der \texttt{AppPlaceHistoryMapper} dem
    History Management und wird u.A. zur Registrierung der Start View benötigt.
    \item \texttt{PlaceControllerProvider.java}\\
    Die Schnittstelle zu den View Places, welche wie erwähnt in
    den \texttt{ViewActivityMappern} aufgerufen werden und somit den
    Browserzugriff auf die View Implementierungen ermöglichen.
    \item \texttt{ProductionGinModule.java}\\
    In dieser Klasse werden die für GIN typischen \texttt{bind}-Befehle
    verwendet. Diese dienen u.A. dazu die View Interfaces an die View
    Implementierungen zu binden sowie die Start View festzulegen.
  \end{itemize}  
  \item View Klassen und Dateien, welche für jede View implementiert werden,
  basierend auf dem MVP Pattern und den UiBindern
    \begin{itemize}
    \item \texttt{'Name'Activity.java}\\
    Diese Klasse implementiert den Presenter und beinhaltet
    eine Instanz der \texttt{View} sowie des \texttt{PlaceController}.
    Durch den \texttt{PlaceController} wird z.B. die Navigation zwischen den Views
    ermöglicht mittels einer darauf angewendeten \texttt{goTo}-Methode, welcher
 	ein \texttt{Place} übergeben wird.
    \item \texttt{'Name'Place.java}\\
    Diese Implementierungen sehen prinzipiell immer gleich aus. Unterschieden
    wird hierbei die dazugehörige \texttt{View}. Über den \texttt{Place} wird
    die Navigation zwischen den Webseiten ermöglicht, wobei der Name des
    \texttt{Place} in der URI Zeile des Browsers steht.
    \item \texttt{'Name'View.java}\\
    Hierbei handelt es sich um ein Interface, welches das Presenter Interface
    definiert und die Oberklasse für die jeweiligen View Implementierungen ist.
    Dadurch wird der einfache View Austausch durch MVP ermöglicht, welches
    zusätzlich über einen \texttt{bind}-Befehl innerhalb des
    \texttt{ProductionGinModule} festgelegt werden muss.
    \item \texttt{'Name'ViewImpl.java}\\
    Die im Browser als Webseite sichtbare View Implementierung beinhaltet eine
    Instanz, des durch der jewiligen \texttt{View} und \texttt{Activity}
    definierten Presenter. Dadurch wird die Kontrolle gemäß MVP abgegeben. Die
    Klasse kann entweder das gesamte GUI erstellen oder mittels UiBinder einen
    Teil der View Komponenten abgeben, wie im Fall von vordefinierten
    Labels.
    \item \texttt{'Name'ViewImpl.ui.xml}\\
    Innherhalb dieser Datei können View Komponenten sowie dessen
    Style-Eigenschaften definiert werden.
  \end{itemize} 
  \item Views und Elemente, die auf jeder Webseite zu sehen
  sind,
    \begin{itemize}
    \item werden gemäß dem MVP Pattern und wie eine View erstellt.
    \item sind innerhalb des \texttt{AppEntryPoint} enthalten und definiert.
    \item implementieren jeweils einen eigenen
    \texttt{ViewActivityMapper}.
  \end{itemize} 
\end{itemize}
% -------------------View Eintragungen--------------------------
Unter Betrachtung, dass ausschließlich eine View Implementierung existiert
und basierend auf dieser Architektur muss zur Erzeugung einer View:\\
\begin{tabular}{ll} 
\addlinespace
ein Eintrag dazu in den &  \quad \quad folgende Klassen und\\
\underline{folgenden Klassen geschehen} & \quad \quad \underline{Dateien erzeugt
werden}
\\
\addlinespace
\addlinespace
  - \texttt{'Name'ActivityMapperImpl.java} & \quad \quad -
  \texttt{'Name'Activity.java}\\
  - \texttt{AppPlaceHistoryMapper.java} & \quad \quad -
  \texttt{'Name'Place.java}\\
  - \texttt{ProductionGinModule.java} & \quad \quad -
  \texttt{'Name'View.java}\\
    	& \quad \quad - \texttt{'Name'ViewImpl.java}\\
    	& \quad \quad - \texttt{'Name'ViewImpl.ui.xml}\\
\addlinespace
\end{tabular}\\
Existieren zu einer \texttt{View} mehrere View Implementierungen so müssen
mehrere Eintragungen innerhalb des \texttt{ProductionGinModule} getätigt und
mehrere \texttt{'Name'ViewImpl.java} und
\texttt{'Name'ViewImpl.ui.xml} erstellt werden. Dadurch wird eine
gute Abstraktion und lose Kopplung geschaffen. Dies ist jedoch sehr
aufwändig und fehleranfällig, da bei der Erstellung einer Vielzahl von
Klassen schnell ein Eintrag vergessen werden kann.

% -------------------Packages--------------------------
Zu den genannten Architekturvorstellungen gehört zusätzlich eine Paketierung,
die auch durch vorherige GWT Projekte entstand. Der Vorteil der folgenden
Gliederung der Pakete besteht darin, dass innerhalb der
\texttt{'Name'.gwt.xml} Konfigurationsdatei das \texttt{source}-Tag, welches
den Pfad für den zu übersetzenden Java Code angibt, wie folgt: \texttt{<source
path='client'/>} definiert werden kann. Folgend die Gliederung der Klassen und
Dateien innerhalb ihrer Packages:\\\\
\begin{tabular}{ll} 
\textbf{Package} &  \quad \quad \textbf{Klassen und Dateien}\\
\addlinespace
\hline
\addlinespace
  \texttt{'projektname'} & \quad \quad \texttt{'Name'.gwt.xml}\\
  \addlinespace
  \hline
  \addlinespace
  \texttt{'projektname'.client} & \quad \quad \texttt{AppEntryPoint.java}\\
  \addlinespace
  \hline
  \addlinespace
  \texttt{'projektname'.client.common} & \quad \quad
  \texttt{AbstractView.java}\\
    	& \quad \quad \texttt{AbstractActivityDefaultImpl.java}\\
    	& \quad \quad \texttt{'Name'ViewActivityMapperImpl.java}\\
    	& \quad \quad \texttt{AppPlaceHistoryMapper.java}\\
   \addlinespace
   \hline
   \addlinespace
  \texttt{'projektname'.client.gin} & \quad \quad
  \texttt{AppGinjector.java}\\
      	& \quad \quad \texttt{PlaceControllerProvider.java}\\
    	& \quad \quad \texttt{ProductionGinModule.java}\\
  \addlinespace
  \hline
  \addlinespace
  \texttt{'projektname'.client.view} & \quad \quad
  \texttt{'Name'Activity.java}\\
  		& \quad \quad \texttt{'Name'Place.java}\\
    	& \quad \quad \texttt{'Name'View.java}\\
    	& \quad \quad \texttt{'Name'ViewImpl.java}\\
    	& \quad \quad \texttt{'Name'ViewImpl.ui.xml}\\
\end{tabular}\\\\
Jedoch soll für einen Entwickler die Möglichkeit bleiben innerhalb des
\texttt{view}-Packages, die View Interfaces und ihre dazugehörigen Klassen in
Packages zu gliedern.
Aus diesem Grund soll an dieser Stelle das \texttt{view}-Package nicht
Generator-seitig tiefer gegliedert werden.

Anhand der beschriebenen Architektur wird ersichtlich, dass die Erstellung eines
GWT Projektes hauptsächlich im Bereich der einmalig vorhanden Dateien und
Klassen und die Erstellung einer View im groben immer gleich ist. Dies bietet
zwar den Vorteil der Vereinheitlichung mehrerer GWT Projekte und gewährleistet
eine gewisse Übersichtlichkeit und kurze Einarbeitungszeit in verschiedenen GWT
Projekten, ist aber wie erwähnt sehr aufwändig und fehleranfällig.
Beispielsweise kann über \glqq{}Copy-Paste\grqq{} viel erstellt und
implementiert werden. Jedoch ist dabei das Risiko erhöht, dass es vergessen wird Einträge abzuändern oder
hinzuzufügen. Darüber hinaus kann es passieren, das Einträge enthalten sind, wie
der einer Bibliothek, welche jedoch nicht mehr benötigt werden und somit nicht
im Build-Path enthalten sind.
Diese Beispiele führen potenziell alle dazu, dass die gesamte GWT Anwendung nicht mehr startet
und die Suche nach dem Fehler erschwert wird, da oftmals viele dieser
Fehler flüchtig geschehen können. 
% ---------------------------------------------
% Profil
% ---------------------------------------------
\section{UML Profil}\label{AufBProfil}
Eines der Hauptziele ist wie erwähnt die erleichterte Erstellung von GWT
Projekten unter Einbezug der durch die Architektur gegebenen Vorteile. Zu
welchen auch der einfache Austausch von View Implementierungen unabhängig vom
Model (des MVP) gehört. Des Weiteren sollen der Aufwand und die
Fehleranfälligkeit bei der Erstellung eines GWT Projektes sowie zur Erstellung von Views (vgl.
Abschnitt \ref{AufBZielArchitektur}) minimiert werden. Diese Ziele
erfordern einen hohen Abstraktionsgrad innerhalb des Profils sowie weiterhin des
M1 Modells.

Deswegen soll eine der ersten Überlegungen dazu führen, dass das
gesamte GWT Projekt auf ein gemeinsames Element reduziert wird. Dieses
Element erscheint im Rahmen des umzusetzenden
MVP Patterns innerhalb der View Klassen und Dateien
\texttt{Activity.java}, \texttt{Place.java},
\texttt{View.java}, \texttt{ViewImpl.java}
und \texttt{ViewImpl.ui.xml}. Jedoch erschien dadurch der Aufwand bei der
Erstellung von Views nicht minimiert, da durch diese Stereotypenbildung, diese
im M1 Modell weiterhin Einsatz finden müssen. Aus diesem Grund ist eine weitere
Suche nach einem gemeinsamen Element erforderlich. Dies führt, durch die sich
auch in diesem Bereich als vorteilhaft herausstellende Architektur, dazu, dass
dieses das unterste Element, die \textit{View Implementierung}, ist. Sie
erscheint ausreichend für die Erstellung der gesamten, einmalig vorhandenen
Klassen und Dateien sowie der View Klassen und Dateien, da innerhalb der View
Implementierung und der simultanen und vereinheitlichten Namensbennenung alle
notwendigen Teile generierbar wären. 

Eine weitere Anforderung ist der Austausch der View Implementierungen durch den
Einsatz von MVP. Dieser kann jedoch nicht ausschließlich durch die View
Implementierungen erfolgen, da hierbei eine Möglichkeit für die
Verbindung mehrerer View Implementierungen gegeben werden muss. Aus diesem Grund
soll das \textit{View Interface} genutzt werden. Dies stellt die Verbindung von mehreren
View Implementierungen in Form einer Oberklasse her. Zusätzlich bietet dies die
Möglichkeit bestimmte Methoden zu definieren, welche für jede implementierende
View nützlich ist. 

View Implementierungen haben zusätzlich View Komponenten und darüber hinaus ist
die Umsetzung einer Navigation bzw. das Verhalten bei Interaktion mit
View Komponenten ein weiteres umzusetzendes Ziel. In vorhergehenden Generator
Projekten ging die Umsetzung dessen mittels Enumerations hervor. Diese sind
sinnvoll wenn eine View Komponente mehrere Attribute z.B. eine \texttt{value}
haben und kein konkreter Nachbau von bestehenden Frameworks erfolgen soll. In dem Profil
wird ein hoher Abstraktionsgrad erwartet, damit eine Vereinfachung gewährleistet
werden kann. Deswegen bildet der Einsatz von Enumerations als Typ für
View Komponenten einen Mehrwert für die Umsetzung des Profils. Zusätzlich wird ein Layouting als
Ziel ausgeschlossen, wodurch die Nachteile der Verwendung von Enumerations
verringert werden. Dadurch ergibt sich zusätzlich der Einsatz von \textit{View
Komponenten} als Stereotypen mit dem Attribut \texttt{type} vom Typ
\textit{Enumeration View Objekt Typ}.

Für Navigationselemente wie Buttons und Umsetzung der Navigation ist eine
Überlegung ein weiteres Profil für ActivityDiagramme und somit ein
ActivityDiagramm als M1 Modell zu nutzen. Dieses Mittel ermöglicht eine
Übersicht über die Navigations- bzw. Verhaltensstruktur einer Anwendung
und ist somit gut geeignet. Dennoch ist das Ziel der Vereinfachung nicht
gegeben, da für jedes Navigationselement ein extra Eintrag in einem
ActivityDiagramm erfolgen muss. Somit entsteht eine Redundanz innerhalb der
verschiedenen M1 Modelle und der Mehrwert der Navigationsgenerierung wird
gemindert aufgrund des Mehraufwandes. Die Verwendung von Enumerations für View
Komponenten ermöglichte die Idee, dass für navigierbare bzw. verhaltensbasierte
Elemente eine ähnliche Handhabung dienlich sein kann. Diese ermöglichen die Erfüllung der
Zielsetzung und bieten über die Angabe eines Attributes \texttt{goTo} die
Navigation sowie über weiterer Attribute z.B. \texttt{openPopup} eine
Generierung verhaltensbezogener Inhalte über das M1 Modell. Dies führt zu der
Stereotypenbildung der \textit{Navigations View Komponenten} mit
\textit{Enumeration Navigations View Objekt Typen}. Diese Enumeration enthält
dann z.B. \texttt{TREEITEM}, \texttt{BUTTON} oder \texttt{MENUITEM}, d.h. View
Komponenten von GWT, welche für Navigation oder Verhaltensspezifikationen vorgesehen sein sollen.

\textit{Navigations View Komponenten} und \textit{View Komponenten} sollen 
Properties sein, da eine \textit{View Implementierung} diese View Komponenten
enthalten kann.

Zur Kopplung von großen View Implementierungen und zur Erweiterung der
vorgegebenen Mittel anhand der Architektur soll eine weitere Klasse als Stereotyp dienen, die
\textit{eigenen View Komponenten}. Darin sollen bestehende View Komponenten
enthalten sein und somit eine \textit{eigene View Komponente}, z.B. eine große
Datentabelle, bilden, die innerhalb mehrerer \textit{View Implementierungen}
enthalten sein kann.

Im Bereich von Views, z.B. ein Header, die auf allen Websites innerhalb des
Browsers sichtbar sind, soll eine weitere Stereotypenbildung stattfinden. Diese Views
sind \textit{permanente Views} und sollen sich zusätzlich in \textit{Header} und
\textit{Footer} gliedern, da diese, durch
ihre Positionierung in einer Webseite (Header oben, Footer unten), konkrete
permanente Views sind. Diese Views verhalten sich simultan zu den normalen Views (vgl. Abschnitt \ref{AufBZielArchitektur}),
weshalb die \textit{permanenten Views} als \textit{View Implementierungen} von
dem \textit{View Interface} umgesetzt werden können.
Dadurch wird zusätzlich der Austausch von permanenten Views gemäß MVP
ermöglicht.
% ---------------------------------------------
% M1 Modell
% ---------------------------------------------
\section{M1 Modell}\label{AufBM1}
Basierend auf dem UML Profil soll das Modell View Implementierungen enthalten,
welche wiederum View Komponenten als Properties beinhalten und eigene
View Objekte zu diesen View Komponenten zugeordnet sein können. Darüber hinaus
sollen über Attribute innerhalb von dem Stereotyp Navigations View Komponenten,
im M1 Modell enthaltene View Implementierungen angegeben werden. Somit kann z.B.
über das Attribut \texttt{goTo} die Navigation zu dieser View Implementierung
durch die Generierung dessen erfolgen.
Weiterhin sollen permanente View Implementierungen in Form z.B. eines
Headers erstellt werden, welche dann auf jeder View innerhalb des Browsers
sichtbar sind. Eine Implementierung eines Headers soll ein Menu enthalten,
das durch MenuItems die Navigation zu verschiedenen Webseiten ermöglicht.
Zusätzlich sollen View Interfaces, welche verschiedene
Implementierungen haben, erstellt werden. Dadurch kann die Generierung des
MVP Patterns getestet werden.
% ---------------------------------------------
% Generator
% ---------------------------------------------
\section{Generator}\label{AufBGenerator}
Im Bereich des UML Profils sowie des M1 Modells sind außer der Views und ihren
Implementierungen und ihren View Komponenten die weiteren Klassen und Dateien
ungeachtet geblieben. Aus diesem Grund muss der Generator so geschrieben werden,
dass aus dem M1 Modell die gesamte Ziel-Architektur generiert werden kann.
Dies soll potenziell auch ermöglicht werden indem ausschließlich eine View
Implementierung erstellt wird, welche ohne Attribute und Methoden ausgestattet ist. Dazu soll
eine Unterleitung erfolgen, welche einerseits die einmalig vorhandenen Klassen
mit Inhalten (auch View abhängig) und andererseits die Views
generiert. Gemäß dem Fall, dass View Interfaces in dem M1 Modell vorhanden sind,
sollen die Klassen \texttt{'Interfacename'View.java},
\texttt{'Interfacename'Activity.java} und \texttt{'Interfacename'Place.java} den Interfacenamen tragen. Die
Implementierungsklassen bzw. Dateien, der View Intefaces,
\texttt{'Klassenname'ViewImpl.java} und \texttt{'Klassenname'ViewImpl.ui.xml}, haben die
Implementierungsklassennamen. Darüber hinaus sollen die viewabhängigen Inhalte,
wenn mehr als eine View Implementierung vorhanden ist, mittels eines
\texttt{boolean} \texttt{binding} dementsprechend unterschieden werden. Dadurch
sollen alle Inhalte basierend auf View Implementierungen mit dem
\texttt{binding} gleich \texttt{false} auskommentiert werden.
Dadurch kann ein einfacher Austausch bei einem Wechsel der View Implementierung
über das MVP Pattern ermöglicht werden, ohne Code hinzufügen zu müssen.
In dem Fall, dass kein View Interface zu einer Implementierung gehört, so werden
alle Klassen, Dateien und View abhängigen Inhalte mittels des
Implementierungsklassennamens generiert.

Für alle permanenten Views gilt ein
ähnliches Vorgehen wie bei den normalen Views. Jedoch haben diese die
Besonderheit, dass sie in jeder View zu sehen sind und somit zusätzlich zu dem
Hauptinhalt bzw. der HauptView in dem \texttt{AppEntryPoint} eingetragen
werden müssen. Dazu erfolgt eine durch das Profil vorgegebene Unterscheidung
zwischen normalen permanenten Views und Header und Footer. Da Header (oben) und Footer
(unten) eine feste Positionierung auf einer Webseite haben, sollen diese durch
den Generator dementsprechend in dem \texttt{AppEntryPoint} positioniert
werden.
Alle anderen vorhandenen permanenten Views sollen dazwischen eingetragen
und müssen später durch den Entwickler im generiertem Code positioniert werden,
da dies layoutspezifisch ist.

Der Entwickler soll weiterhin zusätzliche Änderungen vornehmen müssen. Diese
betreffen die Startwebseite der GWT Webanwendung, welche in dem generiertem
Code anstatt einer konkreten Angabe des Klassennamens, innerhalb des Befehls,
über \glqq{}Start\grqq{} gekennzeichnet wird. Dieses Vorgehen kann das
Verständnis für die Architektur stärken und es werden überflüssige und überfüllende
Attribute innerhalb des UML Profils und M1 Modells vermieden.

Darüber hinaus soll der Generator auch strukturelle und redundanzfreie
Anforderungen erfüllen. Dazu sollen \glqq{}Util\grqq{} Generatoren für u.A.
Package Declarations, Constants und abfragebedingter \texttt{query}-Blöcke,
welche mehrfach Verwendung finden, geschrieben werden. Zusätzlich sollen weitere
Trennungen innerhalb der generierbaren Teile erfolgen. Zu diesen gehören die
Generierung der View Interfaces und die Generierung der konkreten View Implementierungen. Wobei
hierbei nochmals anzumerken ist, dass die Interface Generierung für die View
Implementierungen auf die Generierung der konkreten View Implementierungen
zugreift. Eine weitere Trennung soll innerhalb der einmalig vorhandenen Klassen
und Dateien erfolgen, da manche Klassen und Dateien nicht View abhängige Inhalte
haben.
