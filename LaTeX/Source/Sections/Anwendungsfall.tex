\section{Anwendungsfall}
\label{Anwendungsfall}
Prinzipiell soll es über den Generator möglich sein verschiedene Anwendungsfälle
zu generieren. Beispielhaft soll dementsprechend eine
einfache Homepage mit u.A.
einem Portfolio und einer Newsseite sowie einem Header über dem alle Seiten
erreichbar sind als Anwendungsfall dienen. Weiterhin soll mit zwei
M1 Modellen gearbeitet werden, in dem der Anwendungsfall inhaltlich leicht
abgeändert wird.
Ein M1 Modell dient Testzwecken und das andere der Vervollständigung der
gesamten Homepage mit Änderungen in dem generiertem Code z.B. zur Gestaltung der
Homepage und einer vollständigen und geordneten Projektstruktur.
Dadurch können alle Testfälle und verschiedene Anwendungsfälle abgedeckt werden,
inklusive der Einbindung von View Komponenten, und eine Trennung zwischen notwendigen Änderungen im
generiertem Code, durch ein einheitliches Projekt, erfolgen.