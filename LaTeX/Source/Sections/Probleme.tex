\section{Probleme} \label{Probleme}
Ein umfangreiches Softwareprojekt wie dieses führt unweigerlich eine Reihe unterschiedlichster Probleme herbei. Speziell im Generator sind einige davon aufgetreten.
Eines der größten Probleme war in MTL der Umgang mit Variablen. So war es nicht möglich, trotz etlicher verschiedener Versuche, eine Boolean Variable anzulegen, die nicht final ist. Es wurde beispielsweise mit dem von MTL zur Verfügung gestellten /textit{let[]} Block versucht, unterschiedliche Variablen anzulegen und innerhalb dieses Blockes zu verändern. Dies gestaltete sich als unmöglich, da im Let-BLock definierte Variablen immer final gesetzt werden. Anschließend wurde getestet, ob man mit Hilfe einer Java Klasse (/textit{BooleanHelper.java}), eine einfache Klasse mit einem Getter und Setter für ein Boolean Wert anlegen und abfragen konnte. Die Setter-Methode funktioniert soweit einwandfrei. Das Problem bestand aber darin, diese zuvor gesetzte Variable mittels Getter-Methode wieder auszulesen. Beim Ausführen der Queries kam der Verdacht auf, das sobald die Setter-Methode aufgerufen wurde, dieser Vorgang eine neue Instanz der Klasse erzeugte, welche nicht mehr den zuvor gesetzten Wert gespeichert hat. Als nächsten möglichen Lösungsschritt wurde versucht die Java Klasse als Singleton zu gestalten, aber auch dies führte zu unterschiedlichsten Fehlermeldungen oder gar gänzlicher Verweigerung der Ausführung. 
Dieses Problem konnte somit im zeitlichen Rahmen dieses Projektes vorerst nicht gelöst werden, sodass speziell in der /textit{ui.xml} die doppelten UI-Elemente per Hand entfernt werden müssen.
 
Ein weiteres Problem ist die Redundanz im Generator. Bei der Entwicklung wurde der Fokus auf die Funktionalität des Generators gelegt. Aber durch das doch sehr umfangreiche Projekt, fehlte am Ende die Zeit, eine gründliche Optimierung vorzunehmen, so dass noch einige unnötige Redundanzen auftreten. Dieses erschwert zwar die Lesbarkeit des Generators erschwert, aber seine Funktionalität ist trotzdem gegeben.