\section{GWT}
\label{GWT}
Das Google Web Toolkit, kurz GWT, ist ein open-source Projekt von Google. GWT
ist ein Framework, welches genutzt werden kann um Webanwendungen mittels der
Programmiersprache Java zu implementieren. Dabei �bersetzt GWT den gesamten
Source-Code in JavaScript Code und DOM-Elemente. W�hrend der �bersetzung des
Java Codes zu JavaScript Code werden dar�ber hinaus Optimierungen vorgenommen
wie das L�schen von dead-Code. Dies f�hrt potenziell dazu, dass komplexe
Anwendungen im Browser schneller ausgef�hrt werden k�nnen. Dar�ber hinaus bietet
GWT noch weitere M�glichkeiten, die dem Entwickler einer Webanwendung zu Gute
kommen. Dazu z�hlen u. A. das Integrieren von JavaScript Code oder von
JavaScript Bibliotheken innerhalb des Java Codes durch das JavaScript Native
Interface, kurz JSNI und das sogenannte Code-Splitting, welches einem Entwickler
erm�glicht sogenannte Split Points innerhalb des Codes zu setzen, welche dazu
f�hren, dass bei der Ausf�hrung der Anwendung bestimmte Inhalte ab dem Split
Point sp�ter nachgeladen werden und dadurch die Startladezeit verringern.
Google bietet mit zu den genannten Eigenschaften weitere positive
Software Engineering Aspekte. Durch GIN (GWT INjection) 
